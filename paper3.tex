% \subsection{Verifying Groups in Linear Time}

% Ta članek obravnava temeljni algoritemski problem odločanja, ali podana \(n \times n\) tabela opisuje grupo. Deterministični pristop k tej težavi je že dolgo temeljil na Lightovi ugotovitvi (1949), ki zahteva \(\mathcal{O}(n^2 \log n)\) časa, medtem ko sta Rajagopalan in Schulman (FOCS 1996) s svojo naključno metodo znižala zapletenost v \(\mathcal{O}(n^2 \log(1/\delta))\), ob majhni verjetnosti za napako \(\delta\). Glavni doprinos tega dela je nov deterministični algoritem, ki deluje v \(\mathcal{O}(n^2)\) času, kar je optimalno glede velikost vhodne tabele \(n \times n=n^2\). 
% Ključni izziv je preverjanje asociativnosti: ker je treba za asociativnost preveriti \((a \cdot b) \cdot c = a \cdot (b \cdot c)\) za vse trojice \(a, b, c\), bi naivni postopek obsegal \(\Theta(n^3)\) preverjanj. Delna znana izboljšava uporabi množico generatorjev velikosti \(\log(n)\) (Lightov pristop) in s tem zmanjša število preverjanj, a še vedno pri \(\mathcal{O}(n^2 \log n)\). Novi algoritem to oviro preseže z vpeljavo koncepta "4-asociativnosti" na skrbno izbranem majhnem podmnožičnem sistemu \(S \subseteq G\). Množica \(S\) se imenuje "osnova" (basis), če lahko vsakega elementa grupe \(G\) izrazimo kot produkt dveh elementov iz \(S\). Avtorji dokažejo, da je, če taka osnova izpolnjuje pogoje asociativnosti (posebno 4-asociativnost) za vse četverice iz \(S\), potem asociativnost velja za celotno tabelo (tj. celo grupo). Ker lahko \(|S|\) izberemo reda \(\sqrt{n}\), preverjanje 4-asociativnosti zahteva \(\mathcal{O}(|S|^4) = \mathcal{O}(n^2)\) časa, kar točno sovpada z najboljšim možnim časom, glede na velikost vhoda.
% Sama konstrukcija takšne majhne osnove temelji na globljih grupno-teoretskih idejah. Avtorji izrabijo dejstvo, da ima vsaka končna grupa, ki ni praštevilske velikosti, "veliko podgrupo" (large subgroup) z indeksom največ \(\sqrt{n}\). Prisotnost takih podgrup omogoča učinkovito dekompozicijo grupe v dve podmnožici, katerih velikosti se množita v največ konstanto \(|G|\). S postopnim iskanjem takih velikih podgrup se nato sestavi zahtevana osnova (base). V tem procesu se identificirajo pomembne normalne podgrupe (če jih grupa ima) s postopkom, ki dinamično sledi konjugacijskim razredom in preverja, ali ostaja unija teh razredov zaprta za množenje. Če se izkaže, da grupa ni preprosta, se algoritem posveti ustrezni normalni podgrupi ali komplementarnemu kvocientu in nadaljuje rekurzivno. Če je grupa preprosta, avtorji uporabijo klasifikacijski izrek o končnih preprostih grupah in znane konstrukcije velikih podgrup v tem okviru.
% Ko je osnova enkrat določena, je preverjanje, da obstaja iskani identitetni element in da ima vsak element skupine svoj inverz, enostavno izvedljivo v \(\mathcal{O}(n^2)\) z neposrednim preverjanjem. Ključen korak je tako "čiščenje" dragega preverjanja asociativnosti po celotni tabeli: zadostuje preverjanje 4-asociativnosti na manjši "osnovi". Poleg tega vgradijo še pregled normalnih podgrup in iskanje velikih podgrup, vse skupaj v \(\mathcal{O}(n^2)\) ali \(\mathcal{O}(n^{1+\epsilon})\) pri izvedbi dodatnih dekompozicij. Celota postopkov zagotavlja, da se končno preverjanje, ali vhodna tabela dejansko predstavlja grupo, izvede v \(\mathcal{O}(n^2)\). To pomeni nov deterministični mejnik in dokončno odgovarja na dolgo odprto vprašanje o natančni časovni zahtevnosti preverjanja grupne strukture iz Cayleyjeve tabele.
% % Upam da sem pravilno razumel delovanje algoritma. To se enkart dobro perveri.

Given an \(n \times n\) multiplication table, the task is to decide whether it is a Cayley multiplication table of a group. To determine whether a multiplication table describes a group, you need to test for the identity element, inverses, and associativity. The first two can be trivially checked in \(\mathcal{O}(n^2)\) time. The main challenge lies in efficiently testing associativity. Light's associativity test shows that it suffices to verify associativity for all triples \((a, b, c)\) where \(a, c \in G\) and \(b \in R\), with \(R\) being a generating set for \(G\). A deterministic algorithm can compute a generating set \(R\) of size \(\lfloor \log n \rfloor\) in \(\mathcal{O}(n^2)\) time, leading to an overall \(\mathcal{O}(n^2 \log n)\) algorithm for associativity testing. 
\\
The paper improves this to an optimal \(\mathcal{O}(n^2)\) algorithm by introducing a novel approach based on \emph{4-associativity over a basis}. The key idea is to reduce the problem to verifying a stronger associativity condition on a carefully chosen subset \(S \subseteq G\) of size \(\mathcal{O}(\sqrt{n})\), called a \emph{basis}. If 4-associativity holds for \(S\), then the entire multiplication table is associative. Here a basis for \((G, \cdot)\) is a set \(S \subset G\) such that \(|S| = \mathcal{O}(\sqrt{n})\) and \(S \cdot S = G\) (every element of G can be written as a product of two elements from S) and 4-associativity of \(S\) is defined by requiring that every \(a, b, c, d \in S\) satisfy \(((ab)c)d = (ab)(cd) = (a(bc))d = a((bc)d) = a(b(cd))\).
\\
The construction of this basis relies on finding large subgroups \(H \leq G\) where \(|H| \geq \sqrt{|G|}\). For non-simple groups, this is achieved through recursive decomposition using normal subgroups. For simple groups, the algorithm employs specialized constructions based on the group's type (alternating, Lie type, etc.).
The complete verification flows through these natural reductions: from the full multiplication table to basis testing, then to large subgroup identification, and finally to simple group handling when needed. Each step preserves the \(\mathcal{O}(n^2)\) bound \cite{10756141}. 