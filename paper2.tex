\subsection{Abstract}
The paper describes a significant improvement of a \textit{randomized} (Monte-Carlo) algorithm for computing the 3-edge-connected components of a digraph
with $m$ edges in polylogarithmic time $\widetilde{O} (m^{3/2})$. The algorithm described bests the previous one by being deterministic and 
computable in linear time. 

\vspace*{0.5cm}

\subsection{Preliminaries and primary problem}
This algorithm solves the problem of finding \textbf{3-edge-connected components in directed graphs}, for preliminaries; Let $G = (V,E)$
be a strongly connected directed graph with $|V(G)| =n$ and  $|E(G)| =m$. Generally a set of edges $C \subseteq E$ is a \textbf{cut} if $G \setminus C$ is not
strongly connected i.e. there does not exist a directed path between every pair of vertices, if $|C| =k$ we refer to $C$ as a \textbf{$k$-sized cut} of $G$.
Hence a digraph $G$ is $k$-edge-connected if it has no $(k-1)$ cuts. 

\vspace*{0.2cm}

We say that two vertices $v$ and $w$ are $k$-edge-connected, and we denote this relation by $v \leftrightarrow_k w$, if there are $k$-edge-disjoint 
directed paths from $v$ to $w$ and $k$-edge-disjoint directed paths from $w$ to $v$.
We define a \textbf{$k$-edge-connected component} of a digraph $G$ as a maximal subset $U \subseteq V(G)$ such that $u \leftrightarrow_k v, \forall u,v \in U$.

\subsection{How they achieved this improvement}
The new method relies on a substructure of digraphs, known as \textbf{$2$-connectivity-light graph} (denoted 2CLG). This is because the decomposition of digraphs 
into a collection of 2CLGs exists in linear time, and maintains the $3$-edge-connected components of the original graph. They also define the minimal $2$-in 
and -out sets, which contain vertices with out- or in-degree of 2. Formally, we define both here.

\begin{definition}
    A \textbf{$2$-connectivity-light} graph $G$ is a strongly connected digraph that contains two types of vertices; \textbf{ordinary} and \textbf{auxiliary},
    that satisfy the following conditions:
    \begin{enumerate}
        \item Any two ordinary vertices are $2$-edge-connected,
        \item each auxiliary vertex has an in- or out-degree of 1,
        \item for every vertex $u$ with out-degree $>1$ and every vertex $v$ with in-degree $>1$, there are 2 edge-disjoint paths from $u$ to $v$,
        \item for each auxiliary vertex $v$ with out-degree (resp. in-degree) of one, all paths from $v$ to any vertex in $G$ (resp. from any vertex in $G$ to $v$),
        we have exactly one common edge, the unique out-edge (resp. in-edge).
    \end{enumerate}
\end{definition}

\begin{definition}
    For a strongly connected digraph $G$ we arbitrarily choose a start vertex $s$. For any vertex $v \neq s$ we define $M(v)$ as a \textbf{minimal $2$-in set} that contains $v$, i.e. a minimal set of vertices which contains $v$, does
    not contains $s$, and has two incoming edges from $V(G) \setminus M(v)$, we denote by $M_R(v)$ the analogous sets in $G^R$, which is the reverse graph of 
    $G$.
\end{definition}

The technique is then based on the following proposition and theorem.

\renewcommand{\theproposition}{I.3} 
\begin{proposition}
    Let $G$ be a 2GLC with a fixed ordinary start vertex $s$. Then for any two ordinary vertices $u$ and $v$, we have $v \leftrightarrow_k u$
    if and only if $M(u) = M(v)$ and $M_R(u) = M_R(v)$.
\end{proposition}
\renewcommand{\theproposition}{\arabic{proposition}}

\renewcommand{\thetheorem}{II.5} 
\begin{theorem}
    Let $G$ be a strongly connected digraph. In linear time, we can construct a collection $H_1, ..., H_t$ of 2CLG graphs, such that:
    \begin{itemize}
        \item For every vertex of $G$ there is exactly one graph among $H_1,..., H_t$, that contains it as an ordinary vertex.
        \item Every two vertices $u$ and $v$ of $G$ are $3$-edge-connected if and only if there is an $i \in \{1, ...,t\}$ such that $u$ and $v$
        are 3-edge-connected.
    \end{itemize}
\end{theorem}
\renewcommand{\thetheorem}{\arabic{theorem}}

\noindent
Thus concluding the paper \cite{10756155}.