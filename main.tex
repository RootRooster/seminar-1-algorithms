%%%%%%%%%%%%%%%%%%%%%%%%%%%%%%%%%%%%%%%%%
% Journal Article
% LaTeX Template
% Version 2.0 (February 7, 2023)
%
% This template originates from:
% https://www.LaTeXTemplates.com
%
% Author:
% Vel (vel@latextemplates.com)
%
% License:
% CC BY-NC-SA 4.0 (https://creativecommons.org/licenses/by-nc-sa/4.0/)
%
% NOTE: The bibliography needs to be compiled using the biber engine.
%
%%%%%%%%%%%%%%%%%%%%%%%%%%%%%%%%%%%%%%%%%

%----------------------------------------------------------------------------------------
%	PACKAGES AND OTHER DOCUMENT CONFIGURATIONS
%----------------------------------------------------------------------------------------

\documentclass[
	a4paper, % Paper size, use either a4paper or letterpaper
	10pt, % Default font size, can also use 11pt or 12pt, although this is not recommended
	unnumberedsections, % Comment to enable section numbering
	twoside, % Two side traditional mode where headers and footers change between odd and even pages, comment this option to make them fixed
]{LTJournalArticle}

\addbibresource{./refs.bib} % BibLaTeX bibliography file

\runninghead{IEEE FOCS 2024} % A shortened article title to appear in the running head, leave this command empty for no running head

\footertext{\textit{Nek Zurnal Dokler ne zvemo} (2024) 12:533-684} % Text to appear in the footer, leave this command empty for no footer text

\setcounter{page}{1} % The page number of the first page, set this to a higher number if the article is to be part of an issue or larger work

\usepackage{datetime}
\newdateformat{monthyeardate}{\monthname[\THEMONTH] \THEYEAR} % Set the date format of \today as only Month Year

% The following lines define the Theorem, Lemma, Proposition, Corollary, Proof, Definition and Remark envs
\usepackage{amsthm}
\newtheorem{theorem}{Theorem}
\newtheorem{lemma}{Lemma}
\newtheorem{proposition}{Proposition}
\newtheorem{corollary}{Corollary}

\newtheorem{definition}{Definition}

\theoremstyle{remark}
\newtheorem{remark}{Remark}

%----------------------------------------------------------------------------------------
%	TITLE SECTION
%----------------------------------------------------------------------------------------

\title{Annual IEEE Symposium on Foundations of Computer Science 2024} % Article title, use manual lines breaks (\\) to beautify the layout

% Authors are listed in a comma-separated list with superscript numbers indicating affiliations
% \thanks{} is used for any text that should be placed in a footnote on the first page, such as the corresponding author's email, journal acceptance dates, a copyright/license notice, keywords, etc
\author{%
	Matej Bel\v sak\textsuperscript{\,2}\thanks{Corresponding author: \href{mailto:mb4390@student.uni-lj.si}{mb4390@student.uni-lj.si}\\ Ljubljana, \monthyeardate\today  }, 
	Nik \v Cade\v z\textsuperscript{\,2},
	Klemen Kav\v ci\v c\textsuperscript{\,2}, 
	Toma\v z J. Leonardis\textsuperscript{\,1},\\
	Bor Panger\v si\v c\textsuperscript{\,1}, 
	Marko Rozman\textsuperscript{\,2}, 
	Pia Sotlar\textsuperscript{\,2}, 
	Vanja Stojanovi\' c\textsuperscript{\,1}
}

% Affiliations are output in the \date{} command
\date{
	\footnotesize\textsuperscript{\textbf{1}}Faculty of Mathematics and Physics, University of Ljubljana\\
	\textsuperscript{\textbf{2}}Faculty of Computer and Information Science, University of Ljubljana
}

% Full-width abstract
\renewcommand{\maketitlehookd}{%
	\begin{abstract}
		\noindent 
		We describe and explore the IEEE Annual Symposium on Foundations of Computer Science (FOCS) conference, which covers a wide
		area of theoretical computer science and mathematical foundations. We shortly describe 6 proceedings from the conference
		exploring new breakthroughs in areas such as graph theory, cryptography, compression algorithms etc.
	\end{abstract}
}

%----------------------------------------------------------------------------------------

\begin{document}

\maketitle % Output the title section

%----------------------------------------------------------------------------------------
%	ARTICLE CONTENTS
%----------------------------------------------------------------------------------------

\section{Introduction}

The FOCS conference or IEEE Annual Symposium on Foundations of Computer Science is an academic conference that coveres
a broad range of theoretical computer science. It is sponsered by the IEEE Computer Science Technical Committe on the 
Mathematical Foundations of Computing (TCMF) \cite{ieeefocs}.

The FOCS 2024 took place in Chicago - Voco Chicago Downtown, from October 27-30, 2024 \cite{focs2024}. It covered a variety of topics, for submissions
the following were mentioned \cite{focs2024}:

\begin{itemize}
	\item Communication complexity
	\item Circuit complexity
	\item Average-case algorithms and complexity
	\item High-dimensional algorithms
	\item Online algorithms
	\item Parametrized algorithms
	\item Spectral methods
	\item Streaming algorithms
	\item Randomized algorithms
	\item Cryptography
	\item Computational complexity
	\item Algorithms and data structures
	\item Quantum computing
	\item Foundations of machine learning
	\item Algorithmic coding theory
	\item Sublinear algorithms
	\item Algorithmic graph theory
	\item Continuous optimization
	\item Foundations of fairness, privacy and databases
	\item Pseudorandomness and derandomization
	\item Markov chains
	\item Analysis of Boolean functions
	\item Economics and computation
	\item Combinatorial optimization
	\item Algebraic computation
	\item Approximation algorithms
	\item Parallel and distributed algorithms
	\item Computational learning theory
	\item Computational geometry
	\item Algorithmic game theory
	\item Combinatorics
\end{itemize}

The official welcome message of FOCS 2024 states that nearly 500 papers were submitted, but does not specify the exact number, out of which 133
were accepted and 131 were presented as talks during the event.

In the following three sections we present three curated papers from the conference... (dopisemo na koncu ko vemo katere)

%------------------------------------------------

\section{O(1) Insertion for Random Walk d-ary Cuckoo Hashing up to the Load Threshold}
\subsection{Random walk d-ary cuckoo hashing}
Cuckoo hashing algorithm's basic idea is to resolve collisions by using two hash functions instead of one. When inserting a new object $x$ into the table, if the slot $h_1(x)$ is occupied, the existing object $x'$ is replaced by $x$, and then $x'$ is inserted into slot $h_2(x')$. If the number of iterations exceeds a threshold, the whole table is rehashed with new hash functions. Random Walk $d$-ary Cuckoo Hashing generalizes the idea by using $d$ hash functions and using a random walk to choose the next hash function in case of a collision. Standard cuckoo hashing, equivalent to the $d = 2$ case, has a load threshold of $0.5$, meaning it can use up to $50\%$ of the hash table space. The $d$-ary hashing improves this threshold. For example, the $d = 3$ case has the threshold at approximately $0.9$, while the insertion time increases linearly with $d$. The insertion algorithm guarantees $\mathcal{O}(1)$ lookup and deletion time, as the object can be retrieved by checking its $d$ positions.

\subsection{Insertion time}
This paper shows that for any $d \geq 4$ hashes and load factor $c < c*d$, the expectation of the random walk insertion time is constant. It shows that the expected number of reassignments during insertion does not depend on the size of the hash table $m$, but only on $d$ and $c$. The article uses bipartite graphs as a representation for the hash functions and objects. In the graph, the objects in set $X$ are connected to their $d$ possible locations in the hash table $Y$. In this representation, a valid perfect matching corresponds to a valid assignment of objects to slots. The existence of such a matching is subject to Hall’s Theorem, which states that a perfect matching exists if and only if every subset $W \subseteq X$ is smaller than its neighborhood in $Y$. Neighborhood meaning all nodes connected to at least one vertex in $W$. The paper shows that when the load factor is below the threshold $c*d$, such perfect matchings exist with high probability as the graph exhibits strong expansion properties. The article identifies ``bad'' sets which have few connections to the rest of the graph. In these sets, the walk might get stuck, but the authors prove that the random walk is unlikely to hit such a set in the first $\mathcal{O}(i^{999})$ steps, and that any random walk which avoids it for the first $\mathcal{O}(i^{999})$ steps is likely to finish in $\mathcal{O}(i)$ steps.

\subsection{Super-polynomial tail bounds}
The paper also provides super-polynomial tail bounds on the insertion time, showing that the probability of a walk exceeding $\ell$ steps decays exponentially in $\ell$, with the exponent approaching $1$ as $d$ increases. They relate their findings to previous work on BFS-based insertion algorithms, noting that random walk insertion achieves comparable or better performance without the computational overhead of BFS path searches \cite{10756020}.

%------------------------------------------------

\section{Computing the 3-Edge-Connected Components of Directed Graphs in Linear Time}
\subsection{Abstract}
The paper describes a significant improvement of a \textit{randomized} (Monte-Carlo) algorithm for computing the 3-edge-connected components of a digraph
with $m$ edges in polylogarithmic time $\widetilde{O} (m^{3/2})$. The algorithm described bests the previous one by being deterministic and 
computable in linear time. 

\vspace*{0.5cm}

\subsection{Preliminaries and primary problem}
This algorithm solves the problem of finding \textbf{3-edge-connected components in directed graphs}, for preliminaries; Let $G = (V,E)$
be a strongly connected directed graph with $|V(G)| =n$ and  $|E(G)| =m$. Generally a set of edges $C \subseteq E$ is a \textbf{cut} if $G \setminus C$ is not
strongly connected i.e. there does not exist a directed path between every pair of vertices, if $|C| =k$ we refer to $C$ as a \textbf{$k$-sized cut} of $G$.
Hence a digraph $G$ is $k$-edge-connected if it has no $(k-1)$ cuts. 

\vspace*{0.2cm}

We say that two vertices $v$ and $w$ are $k$-edge-connected, and we denote this relation by $v \leftrightarrow_k w$, if there are $k$-edge-disjoint 
directed paths from $v$ to $w$ and $k$-edge-disjoint directed paths from $w$ to $v$.
We define a \textbf{$k$-edge-connected component} of a digraph $G$ as a maximal subset $U \subseteq V(G)$ such that $u \leftrightarrow_k v, \forall u,v \in U$.

\subsection{How they achieved this improvement}
The authors derive and prove that instead of a randomized polylogarithmic algorithm, a deterministic linear one exists. Their method relies on a substructure
of digraphs, known as \textbf{$2$-connectivity-light graph} (denoted 2CLG). This is because the decomposition of digraphs into a collection of 2CLGs exists
in linear time, and it maintains the $3$-edge-connected components of the original graph. Besides 2CLGs they rely on the definition of minimal
$2$-in and -out sets, which contain vertices with out- or in-degree of 2. Formally, we define both here.

\begin{definition}
    A \textbf{$2$-connectivity-light} graph $G$ is a strongly connected digraph that contains two types of vertices; \textbf{ordinary} and \textbf{auxiliary},
    that satisfy the following conditions:
    \begin{enumerate}
        \item Any two ordinary vertices are $2$-edge-connected,
        \item each auxiliary vertex has an in- or out-degree of 1,
        \item for every vertex $u$ with out-degree $>1$ and every vertex $v$ with in-degree $>1$, there are 2 edge-disjoint paths from $u$ to $v$,
        \item for each auxiliary vertex $v$ with out-degree (resp. in-degree) of one, all paths from $v$ to any vertex in $G$ (resp. from any vertex in $G$ to $v$),
        we have exactly one common edge, the unique out-edge (resp. in-edge).
    \end{enumerate}
\end{definition}

\begin{definition}
    For a strongly connected digraph $G$ we arbitrarily choose a start vertex $s$. For any vertex $v \neq s$ we define $M(v)$ as a \textbf{minimal $2$-in set} that contains $v$, i.e. a minimal set of vertices which contains $v$, does
    not contains $s$, and has two incoming edges from $V(G) \setminus M(v)$, we denote by $M_R(v)$ the analogous sets in $G^R$, which is the reverse graph of 
    $G$, i.e., the graph obtained from $G$ after reversing the orientation of its edges.
\end{definition}

The technique is then based on the following proposition and theorem.

\renewcommand{\theproposition}{I.3} 
\begin{proposition}
    Let $G$ be a 2GLC with a fixed ordinary start vertex $s$. Then for any two ordinary vertices $u$ and $v$, we have $v \leftrightarrow_k u$
    if and only if $M(u) = M(v)$ and $M_R(u) = M_R(v)$.
\end{proposition}
\renewcommand{\theproposition}{\arabic{proposition}}

\renewcommand{\thetheorem}{II.5} 
\begin{theorem}
    Let $G$ be a strongly connected digraph. In linear time, we can construct a collection $H_1, ..., H_t$ of 2CLG graphs, such that:
    \begin{itemize}
        \item For every vertex of $G$ there is exactly one graph among $H_1,..., H_t$, that contains it as an ordinary vertex.
        \item Every two vertices $u$ and $v$ of $G$ are $3$-edge-connected if and only if there is an $i \in \{1, ...,t\}$ such that $u$ and $v$
        are 3-edge-connected.
    \end{itemize}
\end{theorem}
\renewcommand{\thetheorem}{\arabic{theorem}}


%------------------------------------------------

\section{Paper 3}
% \subsection{Verifying Groups in Linear Time}

% Ta članek obravnava temeljni algoritemski problem odločanja, ali podana \(n \times n\) tabela opisuje grupo. Deterministični pristop k tej težavi je že dolgo temeljil na Lightovi ugotovitvi (1949), ki zahteva \(\mathcal{O}(n^2 \log n)\) časa, medtem ko sta Rajagopalan in Schulman (FOCS 1996) s svojo naključno metodo znižala zapletenost v \(\mathcal{O}(n^2 \log(1/\delta))\), ob majhni verjetnosti za napako \(\delta\). Glavni doprinos tega dela je nov deterministični algoritem, ki deluje v \(\mathcal{O}(n^2)\) času, kar je optimalno glede velikost vhodne tabele \(n \times n=n^2\). 
% Ključni izziv je preverjanje asociativnosti: ker je treba za asociativnost preveriti \((a \cdot b) \cdot c = a \cdot (b \cdot c)\) za vse trojice \(a, b, c\), bi naivni postopek obsegal \(\Theta(n^3)\) preverjanj. Delna znana izboljšava uporabi množico generatorjev velikosti \(\log(n)\) (Lightov pristop) in s tem zmanjša število preverjanj, a še vedno pri \(\mathcal{O}(n^2 \log n)\). Novi algoritem to oviro preseže z vpeljavo koncepta "4-asociativnosti" na skrbno izbranem majhnem podmnožičnem sistemu \(S \subseteq G\). Množica \(S\) se imenuje "osnova" (basis), če lahko vsakega elementa grupe \(G\) izrazimo kot produkt dveh elementov iz \(S\). Avtorji dokažejo, da je, če taka osnova izpolnjuje pogoje asociativnosti (posebno 4-asociativnost) za vse četverice iz \(S\), potem asociativnost velja za celotno tabelo (tj. celo grupo). Ker lahko \(|S|\) izberemo reda \(\sqrt{n}\), preverjanje 4-asociativnosti zahteva \(\mathcal{O}(|S|^4) = \mathcal{O}(n^2)\) časa, kar točno sovpada z najboljšim možnim časom, glede na velikost vhoda.
% Sama konstrukcija takšne majhne osnove temelji na globljih grupno-teoretskih idejah. Avtorji izrabijo dejstvo, da ima vsaka končna grupa, ki ni praštevilske velikosti, "veliko podgrupo" (large subgroup) z indeksom največ \(\sqrt{n}\). Prisotnost takih podgrup omogoča učinkovito dekompozicijo grupe v dve podmnožici, katerih velikosti se množita v največ konstanto \(|G|\). S postopnim iskanjem takih velikih podgrup se nato sestavi zahtevana osnova (base). V tem procesu se identificirajo pomembne normalne podgrupe (če jih grupa ima) s postopkom, ki dinamično sledi konjugacijskim razredom in preverja, ali ostaja unija teh razredov zaprta za množenje. Če se izkaže, da grupa ni preprosta, se algoritem posveti ustrezni normalni podgrupi ali komplementarnemu kvocientu in nadaljuje rekurzivno. Če je grupa preprosta, avtorji uporabijo klasifikacijski izrek o končnih preprostih grupah in znane konstrukcije velikih podgrup v tem okviru.
% Ko je osnova enkrat določena, je preverjanje, da obstaja iskani identitetni element in da ima vsak element skupine svoj inverz, enostavno izvedljivo v \(\mathcal{O}(n^2)\) z neposrednim preverjanjem. Ključen korak je tako "čiščenje" dragega preverjanja asociativnosti po celotni tabeli: zadostuje preverjanje 4-asociativnosti na manjši "osnovi". Poleg tega vgradijo še pregled normalnih podgrup in iskanje velikih podgrup, vse skupaj v \(\mathcal{O}(n^2)\) ali \(\mathcal{O}(n^{1+\epsilon})\) pri izvedbi dodatnih dekompozicij. Celota postopkov zagotavlja, da se končno preverjanje, ali vhodna tabela dejansko predstavlja grupo, izvede v \(\mathcal{O}(n^2)\). To pomeni nov deterministični mejnik in dokončno odgovarja na dolgo odprto vprašanje o natančni časovni zahtevnosti preverjanja grupne strukture iz Cayleyjeve tabele.
% % Upam da sem pravilno razumel delovanje algoritma. To se enkart dobro perveri.

Given an \(n \times n\) multiplication table, the task is to decide whether it is a Cayley multiplication table of a group. To determine whether a multiplication table describes a group, you need to test for the identity element, inverses, and associativity. The first two can be trivially checked in \(\mathcal{O}(n^2)\) time. The main challenge lies in efficiently testing associativity. Light's associativity test shows that it suffices to verify associativity for all triples \((a, b, c)\) where \(a, c \in G\) and \(b \in R\), with \(R\) being a generating set for \(G\). A deterministic algorithm can compute a generating set \(R\) of size \(\lfloor \log n \rfloor\) in \(\mathcal{O}(n^2)\) time, leading to an overall \(\mathcal{O}(n^2 \log n)\) algorithm for associativity testing. 
\\
The paper improves this to an optimal \(\mathcal{O}(n^2)\) algorithm by introducing a novel approach based on \emph{4-associativity over a basis}. The key idea is to reduce the problem to verifying a stronger associativity condition on a carefully chosen subset \(S \subseteq G\) of size \(\mathcal{O}(\sqrt{n})\), called a \emph{basis}. If 4-associativity holds for \(S\), then the entire multiplication table is associative. Here a basis for \((G, \cdot)\) is a set \(S \subset G\) such that \(|S| = \mathcal{O}(\sqrt{n})\) and \(S \cdot S = G\) (every element of G can be written as a product of two elements from S) and 4-associativity of \(S\) is defined by requiring that every \(a, b, c, d \in S\) satisfy \(((ab)c)d = (ab)(cd) = (a(bc))d = a((bc)d) = a(b(cd))\).
\\
The construction of this basis relies on finding large subgroups \(H \leq G\) where \(|H| \geq \sqrt{|G|}\). For non-simple groups, this is achieved through recursive decomposition using normal subgroups. For simple groups, the algorithm employs specialized constructions based on the group's type (alternating, Lie type, etc.).
The complete verification flows through these natural reductions: from the full multiplication table to basis testing, then to large subgroup identification, and finally to simple group handling when needed. Each step preserves the \(\mathcal{O}(n^2)\) bound \cite{10756141}. 

%------------------------------------------------

\section{Lempel-Ziv (LZ77) Factorization in Sublinear Time}
Lempel–Ziv (LZ77) factorization is a string processing technique and the main component of most data compression algorithms, such as ZIP, PDF, and PNG. It separates a given string greedily from left to right into phrases \break \( T = f_1 f_2 \cdots f_z \), so that each phrase is either the first occurrence of a character or the longest prefix of the remaining suffix that has already appeared earlier in the text. Each phrase is encoded either as a that same character or as a pair \((l, i)\), where \(l\) is the length of the phrase and \(i\) is the position of its earlier occurrence. For example, for the string \( T = b \cdot b \cdot a \cdot ba \cdot aba \cdot bababa \cdot ababa \), the LZ77 representation is \((0, b), (1,1), (0, a), (2,2), (3,3), (6, 7), (5, 10)\).

In the RAM model, the theoretical lower bound for LZ77 factorization is \( O(n / \log_{\sigma} n) \), which matches the maximum number of phrases. Despite this, no algorithm achieved this optimal bound until recently.
Earlier algorithms were based on suffix trees or suffix arrays, achieving \( O(n \log \sigma) \) time and \( O(n) \) space. Later improvements reduced space usage to the optimal \( O(n / \log_{\sigma} n) \), but the algorithm still required \(\Omega (n)\) time in the worst case.

Kempa and Kociumaka introduced the first sublinear-time algorithm for LZ77 factorization, running in \( O(n / \sqrt{\log n}) \) time for binary alphabets and \break \( O((n \log \sigma) / \sqrt{\log n}) \) time for larger alphabets, while using optimal \( O(n / \log_{\sigma} n) \) space. They developed a novel index structure that can be built in sublinear time and efficiently finds the leftmost previous occurrence of a substring. This index makes the computation of the Longest Previous Factor (LPF) for each position in the string fast and answers substring occurrence queries in \( O(\log^{\varepsilon} n) \) time.

Instead of relying on the classical method based on Range Minimum Queries (RMQ) over the suffix array—which is too slow for sublinear-time construction—they use a sampling-based approach. The idea is that instead of looking through all the text positions they select a small, representative sample (S) of positions in non-periodic regions (size \(O(n / \log_{\sigma} n)\)). The rule is that if two substrings are equal their starting positions are either both included or both excluded from S.

To check if a substring has appeared before, the problem is turned into a special kind of Range Minimum Query (RMQ). Prefix RMQ finds the smallest value in a range, but only if the prefix matches a given pattern. The authors create a fast and memory-efficient data structure to perform these prefix RMQ queries.

For periodic regions, the algorithm uses two different approaches. If the pattern is only partially periodic, it uses sorted runs and special range queries (3-sided RMQ) to handle them. If the pattern is fully periodic, the algorithm marks the starting positions with a bitvector (a sequence of 0 and 1, where 1 defines the starting postions of the pattern). It then uses fast rank/select queries on this bitvector to quickly find the relevant patterns.

So by using this sampling-based approach with prefix RMQ structure, the authors achive an index that maintains optimal \( O(n / \log_{\sigma} n) \) space complexity, sublinear preprocessing time, and efficient query performance.


%------------------------------------------------

\section{Tight bounds for Classical Open Addressing}
% Članek: Tight bounds for Classical Open Addressing
% Avorji: Micheal A. Bender, William Kuszmaul, Renfei Zhou

The authors investigate the optimal tradeoff between time and space in open-addressed hash tables with a high load factor of $1 - \epsilon$. They introduce the \textit{Rainbow Hash Table}, which achieves $O(1)$ expected query time and $O(\log\log(1/\epsilon))$ update time, while maintaining near-full hash table. They prove this tradeoff is optimal, demonstrating that no open-addressed hash table can achieve better time complexity while supporting such a high load factor.

Paper first introduced \textit{Rainbow Cell}, consisting of $n^{1/4}$ buckets, each containing $n^{3/4}$ slots of which $n^{1/2}$ slots are special \textit{sky slots}.
Rainbow Cell is specialized hash table that operates at load factor of 1 with $O(1)$
time complexity for insertions and deletions, while queries have $O(n^{3/4})$ time
complexity. Each element is randomly assigned either \textit{heavy} (high probability) or \textit{light} (low probability) state with status hash function. Heavy keys are stored only in bucket that it was hashed to. Light keys are stored only in sky slots and can be moved between buckets without disruption structure.

The idea is to reduce problem to subproblems and \textit{Rainbow Hash Table} is introduced. It is a tree structure where each subproblem is implemented as a Rainbow Cell. Elements are randomly assigned a color, which is used to help locate level of where element is going to end up. The probability distribution of colors ensures that most elements are placed in lower levels. Meaning, at lower levels of tree, subproblems become significantly smaller and tree becomes wider. 

Next, Rainbow Hash Table is extended to support dynamic resizing, meaning the hash table could increase and decrease in size while always remaining full (load factor of 1) and preserving its efficiency. Expected query time is $O(1)$ and expected update time of $O(\log\log n)$, where $n$ is current size of hash table. 

The final system supports a load factor up to $1 - \epsilon$ and ensures insertions and deletions remain efficient. The main result of the paper is that Rainbow Hash Table achieves optimal time complexity for open-addressed hash tables while supporting a high load factor of $1 - \epsilon$. This result resolves the open question of whether it is possible to achieve operations with $O(1/\epsilon)$ time complexity for insertions, deletions, and queries simultaneously in an open-addressed hash table.

%------------------------------------------------

\section{Minor Containment and Disjoint Paths in almost-linear time}

The paper \textit{Minor Containment and Disjoint Paths in Almost-Linear Time} by Tuukka Korhonen, Michał Pilipczuk, and Giannos Stamoulis introduces an advanced algorithm that determines whether a graph $H$ is a minor of another graph $G$ in 
$\mathcal{O}_H(n^{1+o(1)})$ time, where $n$ represents the number of vertices in $G$. This algorithm is a further development on the work previously done by o Kawarabayashi, Kobayashi, and Reed which had complexity quadratic relative to the number $n$ of vertices of graph $G$.

\subsection{Minor containment problem}
In graph theory, a graph $H$ is considered a minor of a graph 
$G$ if $H$ can be derived from $G$ through a series of vertex deletions, edge deletions, and edge contractions. The ability to test for minor containment efficiently is crucial, as it underpins the recognition of minor-closed graph classes—collections of graphs where any minor of a member graph is also included in the class. The Graph Minor Theorem (Robertson–Seymour) asserts that such classes can be characterized by a finite set of forbidden minors, making efficient minor testing algorithms highly valuable.
\\

Beyond unrooted minor containment, the authors also address the rooted version of the problem. Here, $G$ is provided with a set of roots $X\subseteq V(G)$, and certain branch sets of the desired minor model of
$H$ must include specified subsets of $X$. The proposed algorithm solves this in $\mathcal{O}_{H,|X|}(m^{1+o(1)})$ time, where $m$ denotes the total number of vertices and edges in $G$. This encompasses the Disjoint Paths problem, a classic challenge in graph algorithms that seeks paths between specified pairs of terminals without shared vertices or edges. The new algorithm achieves this in $\mathcal{O}_k(m^{1+o(1)})$ time, with $k$ being the number of terminal pairs. 

\subsection{Main Contributions}
The efficiency of the proposed algorithm of the authors comes from two primary components:

\begin{itemize}
    \item 
    Implementation of the Irrelevant Vertex Technique: Utilizing the \textit{dynamic treewidth} data structure developed by Korhonen et al., the algorithm applies the irrelevant vertex technique of Robertson and Seymour in almost-linear time on apex-minor-free graphs. This technique identifies and removes vertices that do not contribute to the minor structure, simplifying the graph without affecting the minor relationship.
\item Application of Recursive Understanding: Leveraging recent advancements in almost-linear time flow/cut algorithms, the recursive understanding technique effectively reduces complex graphs to apex-minor-free graphs. This reduction simplifies the problem, making it more tractable for the minor containment test.
\end{itemize}

\subsection{Conclusion}
These innovations collectively enable the algorithm to perform minor containment and solve the Disjoint Paths problem with better efficiency. In the paper the authors provide detailed proofs for theoretical advancements and complexity analyses, ensuring the robustness of their approach.

The authors also believe that the key new insights of the paper - namely an the almost-linear time implementation of recursive understanding and the almost-linear time implementation of the irrelevant vertex rule on apex-minor-free graphs with the dynamic treewidth data structure  could find applicability much more broadly in the context of fixed-parameter algorithms and computational problems in the theory of graph minors.






%----------------------------------------------------------------------------------------
%	 REFERENCES
%----------------------------------------------------------------------------------------

\printbibliography % Output the bibliography

%----------------------------------------------------------------------------------------

\end{document}
