%%%%%%%%%%%%%%%%%%%%%%%%%%%%%%%%%%%%%%%%%
% Journal Article
% LaTeX Template
% Version 2.0 (February 7, 2023)
%
% This template originates from:
% https://www.LaTeXTemplates.com
%
% Author:
% Vel (vel@latextemplates.com)
%
% License:
% CC BY-NC-SA 4.0 (https://creativecommons.org/licenses/by-nc-sa/4.0/)
%
% NOTE: The bibliography needs to be compiled using the biber engine.
%
%%%%%%%%%%%%%%%%%%%%%%%%%%%%%%%%%%%%%%%%%

%----------------------------------------------------------------------------------------
%	PACKAGES AND OTHER DOCUMENT CONFIGURATIONS
%----------------------------------------------------------------------------------------

\documentclass[
	a4paper, % Paper size, use either a4paper or letterpaper
	10pt, % Default font size, can also use 11pt or 12pt, although this is not recommended
	unnumberedsections, % Comment to enable section numbering
	twoside, % Two side traditional mode where headers and footers change between odd and even pages, comment this option to make them fixed
]{LTJournalArticle}

\addbibresource{./refs.bib} % BibLaTeX bibliography file

\runninghead{Shortened Running Article Title} % A shortened article title to appear in the running head, leave this command empty for no running head

\footertext{\textit{Journal of Biological Sampling} (2024) 12:533-684} % Text to appear in the footer, leave this command empty for no footer text

\setcounter{page}{1} % The page number of the first page, set this to a higher number if the article is to be part of an issue or larger work

\usepackage{datetime}
\newdateformat{monthyeardate}{\monthname[\THEMONTH] \THEYEAR} % Set the date format of \today as only Month Year

%----------------------------------------------------------------------------------------
%	TITLE SECTION
%----------------------------------------------------------------------------------------

\title{Annual IEEE Symposium on Foundations of Computer Science 2024} % Article title, use manual lines breaks (\\) to beautify the layout

% Authors are listed in a comma-separated list with superscript numbers indicating affiliations
% \thanks{} is used for any text that should be placed in a footnote on the first page, such as the corresponding author's email, journal acceptance dates, a copyright/license notice, keywords, etc
\author{%
	Matej Bel\v sak\textsuperscript{\,2}\thanks{Corresponding author: \href{mailto:mb4390@student.uni-lj.si}{mb4390@student.uni-lj.si}\\ Ljubljana, \monthyeardate\today  }, 
	Nik \v Cade\v z\textsuperscript{\,2},
	Klemen Kav\v ci\v c\textsuperscript{\,2}, 
	Toma\v z Leonardis\textsuperscript{\,1},\\
	Bor Panger\v si\v c\textsuperscript{\,1}, 
	Marko Rozman\textsuperscript{\,2}, 
	Pia Sotlar\textsuperscript{\,2}, 
	Vanja Stojanovi\' c\textsuperscript{\,1}
}

% Affiliations are output in the \date{} command
\date{
	\footnotesize\textsuperscript{\textbf{1}}Faculty of Mathematics and Physics, University of Ljubljana\\
	\textsuperscript{\textbf{2}}Faculty of Computer and Information Science, University of Ljubljana
}

% Full-width abstract
\renewcommand{\maketitlehookd}{%
	\begin{abstract}
		\noindent 
		We will write this at the end
	\end{abstract}
}

%----------------------------------------------------------------------------------------

\begin{document}

\maketitle % Output the title section

%----------------------------------------------------------------------------------------
%	ARTICLE CONTENTS
%----------------------------------------------------------------------------------------

\section{Introduction}

The FOCS conference or IEEE Annual Symposium on Foundations of Computer Science is an academic conference that coveres
a broad range of theoretical computer science. It is sponsered by the IEEE Computer Science Technical Committe on the 
Mathematical Foundations of Computing (TCMF).

The FOCS 2024 took place in Chicago - Voco Chicago Downtown, from October 27-30, 2024. It covered a variety of topics, for submissions
the following were mentioned:

\begin{itemize}
	\item Communication complexity
	\item Circuit complexity
	\item Average-case algorithms and complexity
	\item High-dimensional algorithms
	\item Online algorithms
	\item Parametrized algorithms
	\item Spectral methods
	\item Streaming algorithms
	\item Randomized algorithms
	\item Cryptography
	\item Computational complexity
	\item Algorithms and data structures
	\item Quantum computing
	\item Foundations of machine learning
	\item Algorithmic coding theory
	\item Sublinear algorithms
	\item Algorithmic graph theory
	\item Continuous optimization
	\item Foundations of fairness, privacy and databases
	\item Pseudorandomness and derandomization
	\item Markov chains
	\item Analysis of Boolean functions
	\item Economics and computation
	\item Combinatorial optimization
	\item Algebraic computation
	\item Approximation algorithms
	\item Parallel and distributed algorithms
	\item Computational learning theory
	\item Computational geometry
	\item Algorithmic game theory
	\item Combinatorics
\end{itemize}

The official welcome message of FOCS 2024 states that nearly 500 papers were submitted, but does not specify the exact number, out of which 133
were accepted and 131 were presented as lectures during the event.

In the following three sections we present three curated papers from the conference... (dopisemo na koncu ko vemo katere)

%------------------------------------------------

\section{Paper 1}
Hi, I'm paper 1!

%------------------------------------------------

\section{Paper 2}
Hi, I'm paper 2!


%------------------------------------------------

\section{Paper 3}
\subsection{Random walk d-ary cuckoo hashing}
Cuckoo hashing algorithm's basic idea is to resolve collisions by using two hash functions instead of one. When inserting a new object $x$ into the table, if the slot $h_1(x)$ is occupied, the existing object $x'$ is replaced by $x$, and then $x'$ is inserted into slot $h_2(x')$. If the number of iterations exceeds a threshold, the whole table is rehashed with new hash functions. Random Walk $d$-ary Cuckoo Hashing generalizes the idea by using $d$ hash functions and using a random walk to choose the next hash function in case of a collision. Standard cuckoo hashing, equivalent to the $d = 2$ case, has a load threshold of $0.5$, meaning it can use up to $50\%$ of the hash table space. The $d$-ary hashing improves this threshold. For example, the $d = 3$ case has the threshold at approximately $0.9$, while the insertion time increases linearly with $d$. The insertion algorithm guarantees $\mathcal{O}(1)$ lookup and deletion time, as the object can be retrieved by checking its $d$ positions.

\subsection{Insertion time}
This paper shows that for any $d \geq 4$ hashes and load factor $c < c*d$, the expectation of the random walk insertion time is constant. It shows that the expected number of reassignments during insertion does not depend on the size of the hash table $m$, but only on $d$ and $c$. The article uses bipartite graphs as a representation for the hash functions and objects. In the graph, the objects in set $X$ are connected to their $d$ possible locations in the hash table $Y$. In this representation, a valid perfect matching corresponds to a valid assignment of objects to slots. The existence of such a matching is subject to Hall’s Theorem, which states that a perfect matching exists if and only if every subset $W \subseteq X$ is smaller than its neighborhood in $Y$. Neighborhood meaning all nodes connected to at least one vertex in $W$. The paper shows that when the load factor is below the threshold $c*d$, such perfect matchings exist with high probability as the graph exhibits strong expansion properties. The article identifies ``bad'' sets which have few connections to the rest of the graph. In these sets, the walk might get stuck, but the authors prove that the random walk is unlikely to hit such a set in the first $\mathcal{O}(i^{999})$ steps, and that any random walk which avoids it for the first $\mathcal{O}(i^{999})$ steps is likely to finish in $\mathcal{O}(i)$ steps.

\subsection{Super-polynomial tail bounds}
The paper also provides super-polynomial tail bounds on the insertion time, showing that the probability of a walk exceeding $\ell$ steps decays exponentially in $\ell$, with the exponent approaching $1$ as $d$ increases. They relate their findings to previous work on BFS-based insertion algorithms, noting that random walk insertion achieves comparable or better performance without the computational overhead of BFS path searches.


%----------------------------------------------------------------------------------------
%	 REFERENCES
%----------------------------------------------------------------------------------------

\printbibliography % Output the bibliography

%----------------------------------------------------------------------------------------

\end{document}
